\begingroup
\frontmatter

% Title
\title{Umelá inteligencia \\ Eulerov kôň}
\author{Jakub Hvolka}
\maketitle
\newpage

\tableofcontents

\newpage

\mainmatter


\chapter{Zadanie}\label{ch:zadanie}

Úlohou bolo vytvoriť program, ktorý nájde riešenia na problém známy ako Knight's Tour (ang.).

Kôň začína na šachovnici na vybratom políčku, a úlohou je nájsť postupnosť ťahov tak, aby kôň prešiel celú
šachovnicu a každé políčko navštívil práve raz.

Na riešenie tohto programu som mal implementovať algoritmus \em{depth first search}, ktorý som mal obmedziť buď
časovo, alebo maximálnym počtom výpočtových krokov.
Takto som mal nájsť 10 riešení.
5 pre šachovnicu s rozmermi 5x5 a 5 pre šachovnicu s rozmermi 6x6.
Pre oba rozmery šachovnice som mal nájsť 1 riešenie pri začiatku v rohu šachovnice a 4 ďalšie s náhodným začiatkom.


\chapter{Opis riešenia}\label{ch:opis-riešenia}

Program som písal v jazyku c++.
Rozdelil som ho na dve časti: šachovnica a DFS algoritmus.


\section{Šachovnica}\label{sec:šachovnica}

Šachovnica je uložená v triede \em{ChessBoard}.
Táto trieda obsahuje dáta o šachovnici a jej aktuálnom stave, zoznam riešení ktoré boli nájdené
a postupnosti krokov potrebné na zopakovanie týchto riešení.

\subsection{Metódy}\label{subsec:metódy}

\paragraph{Konštruktor} vytvára prázdnu šachovnicu s koňom na začiatočnom
políčku.

\paragraph{setPos} Presúva koňa na konkrétne políčko podľa vstupu.
Kontroluje, či sa šachovnica nenachádza vo vyriešenom stave.
Ak áno, riešenie sa uloží.

\paragraph{resetPos} vráti späť posledný vykonaný ťah.

\paragraph{operator <<} preťažuje operátor na výstup.
Vypíše report o šachovnici, počte nájdených riešení a riešenia samotné.

Pre účely čitateľnosti riešenia som sa rozhodol vypisovať len šachovnicu
s očíslovanými políčkami, podľa poradia návštevy.
Hodnoty sú opísané v tabuľke~\ref{tab:tabuľka-hodnôt}.
Postupnosť krokov by sa triviálne dala získať z rozdielov medzi jednotlivými ťahmi,
uloženými v solved\_move\_list~\ref{par:solved-move-list}.

\paragraph{get\_last\_pos} vracia aktuálnu pozíciu na ktorej sa kôň nachádza

\paragraph{get\_start\_pos} vracia pozíciu, na ktorej kôň začal.

\subsection{Dáta}\label{subsec:dáta}
Len najdôležitejšie:

\paragraph{board}\label{par:board} ukladá stav šachovnice ako 2D pole celých čísel.
\begin{table}[!ht]
    \centering
    \caption{Tabuľka hodnôt}
    \label{tab:tabuľka-hodnôt}
    \begin{tabular}{cc}
        \toprule
        hodnota            & význam políčka           \\ \midrule
        0                  & nenavštívené             \\
        1                  & počiatočné políčko       \\
        2 až počet políčok & poradie návštevy políčka \\ \bottomrule
    \end{tabular}
\end{table}

\paragraph{solved\_move\_list}\label{par:solved-move-list} zoznam postupností
krokov ktoré vedú k riešeniam.

\paragraph{solved\_boards} zoznam šachovníc vo vyriešenom stave.


\section{Algoritmus}\label{sec:algoritmus}

Hlavná časť programu je implementácia depth first search algoritmu.
Tento algoritmus prehľadáva stavový priestor smerom do hĺbky a následne sa vracia
ak sa kôň dostane na koniec cesty (či už nájde riešenie alebo nie).
Algoritmus sa ukončí ak:
\begin{itemize}
    \item je nájdený požadovaný počet riešení,
    \item je prehľadaný maximálny povolený počet stavov, alebo
    \item je prehľadaný celý stavový priestor.
\end{itemize}

Algoritmus pre každý stav nájde susedné stavy a rekurzívne ich prehľadáva.
Jednotlivé stavy nie sú celé uložené v pamäti, ale len informácia potrebná na
ich návrat do rodičovského stavu je uložená.


\section{Testovanie}\label{sec:testovanie}

\subsection{Spôsob testovania}\label{subsec:spôsob-testovania}

Meral som časy vykonávania algoritmu a pamäť využitú programom.
Merania boli ohraničené maximálnym počtom stavov, ktoré algoritmus mohol navštíviť.

\subsection{Výsledky meraní}\label{subsec:výsledky}

Pri meraní vyšli priemerné časy pre prehľadanie jedného stavu rovnaké pre všetky vstupy, a to v rozmedzí medzi 13 až 15 ns.

Pre šachovnice 5x5 bol algoritmus schopný prejsť všetky stavy (1829420), v prípade, že riešenie neexistovalo.







\section{Záver}\label{sec:záver}







\endgroup